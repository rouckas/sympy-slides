\documentclass[aspectratio=169,xcolor=svgnames]{beamer}
\usetheme{Torino}

\usepackage{epsfig} %for figures
\usepackage{xcolor} %for color
\usepackage[utf8]{inputenc}
\usepackage{multicol}
\usepackage{hyperref}

% latex definitions:
\def\d{{\rm d}}
\def\half{{\textstyle{1\over2}}}



\title[SymPy\hspace{4em}\insertframenumber/
\inserttotalframenumber]{~\\ Symbolic computing with SymPy \\~}


\author[Š. Roučka]
{Štěpán Roučka}

\pgfdeclareimage[height=1.5cm]{mylogo}{sympy-250px}
\institute{\pgfuseimage{mylogo}}

\date{October 18, 2018}

\begin{document}

\begin{frame}
  \maketitle
\begin{center}
\normalsize Materials will be available at \url{http://github.com/rouckas/sympy-slides/}
\end{center}
\end{frame}


\begin{frame}{What is SymPy}
  \begin{block}{Goal}
    Provide a symbolic manipulation library in Python.
  \end{block}
  \begin{block}

    ``SymPy is an open source Python library for symbolic mathematics. It aims to
    become a full-featured computer algebra system (CAS) while keeping the code as
    simple as possible in order to be comprehensible and easily extensible. SymPy
      is written entirely in Python and has only one dependence (mpmath).''

  \end{block}
\end{frame}

\begin{frame}{Past and present}
  \begin{block}{History}
    \begin{itemize}
    \item Ondřej Čertík started the project in 2005.
    \item Development took off in 2007 when SymPy first participated in Google
      Summer of Code. We have participated in every Google Summer of Code since.
    \item In 2011, Aaron Meurer (who also joined from Google Summer of Code) took
      over as lead developer.
    \end{itemize}
  \end{block}
  \begin{block}{Current Status}
    \begin{itemize}
    \item Over 750 contributors.
    \item Current code base has over 500,000 lines of code and documentation.
    \end{itemize}
  \end{block}
\end{frame}

\begin{frame}{Why SymPy?}
    \begin{columns}
        \column{0.5\textwidth}
  \begin{block}{}
    \begin{itemize}
      \item Standalone
      \item Full featured
      \item BSD licensed
      \item Embraces Python
      \item Usable as a library
    \end{itemize}
  \end{block}
        \column{0.5\textwidth}
  \begin{block}{Why SymPy for data scientists?}
      It is up to you
  \end{block}
  \begin{block}{Research Articles citing SymPy}
    \begin{itemize}
        \item \href{https://scholar.google.cz/scholar?cites=2023909674275426142&as_sdt=2005&sciodt=0,5&hl=en}{Google Scholar list}
        \item \href{https://www.zotero.org/groups/525293/sympy/items/}{Zotero group}
        \item \href{https://peerj.com/articles/cs-103/}{Sympy paper}
    \end{itemize}
  \end{block}
    \end{columns}
\end{frame}
\begin{frame}{Projects Using SymPy}
    \begin{itemize}
\item
  \href{http://www.sagemath.org/}{\textbf{Sage}}: A CAS, which includes many open source mathematical libraries.
\item
  \href{https://cloud.sagemath.com}{\textbf{SageMathCloud}}:
  A web-based cloud computing and course management
  platform for computational mathematics.
\item
  \href{http://mathpix.com/}{\textbf{Mathpix}}: An iOS App, that detects
  handwritten math as input, and uses SymPy to evaluate it
  and generate the relevant steps to solve the problem.
\item
  \href{http://www.pydy.org/}{\textbf{PyDy}}: Multibody Dynamics with
  Python.
\item
  \href{http://openrave.org/docs/0.8.2/openravepy/ikfast/}{\textbf{IKFast}}:
  IKFast is a robot kinematics compiler provided by
  \href{http://openrave.org/}{OpenRAVE}. It analytically solves robot inverse
  kinematics equations and generates optimized C++. 
\item
  \href{http://octave.sourceforge.net/symbolic/}{\textbf{Octave Symbolic}}:
  Symbolic package adding symbolic calculation features
  to GNU Octave.
\item
  \href{https://github.com/brombo/galgebra}{\textbf{galgebra}}:
  Geometric algebra (previously \texttt{sympy.galgebra}).
    \end{itemize}
\end{frame}
\begin{frame}{Projects Using SymPy}
\begin{itemize}
\item
  \href{https://github.com/jverzani/SymPy.jl}{\textbf{SymPy.jl}}:
  Provides a Julia interface to SymPy using PyCall.
\item
  \href{https://mathics.github.io/}{\textbf{Mathics}}: A
  free, general-purpose online CAS featuring Mathematica compatible
  syntax and functions.
\item
  \href{http://sfepy.org/}{\textbf{SfePy}}: Simple finite elements in
  Python.
\item
  \href{http://quameon.sourceforge.net/}{\textbf{Quameon}}: Quantum
  Monte Carlo in Python.
\item
  \href{http://lcapy.elec.canterbury.ac.nz/}{\textbf{Lcapy}}:
  Experimental Python package for teaching linear circuit analysis.
\item
  \href{http://digitalcommons.calpoly.edu/cgi/viewcontent.cgi?article=1072\&context=physsp/}{\textbf{Quantum
  Programming in Python}}: Quantum 1D Simple Harmonic Oscillator and
  Quantum Mapping Gate.
\item
  \href{http://mech.fsv.cvut.cz/~stransky/software/latexexpr/doc/}{\textbf{LaTeX
  Expression project}}: Easy \LaTeX{} typesetting of algebraic expressions
  in symbolic form with automatic substitution and result computation.
\item
  \href{https://www.researchgate.net/publication/260585491_Symbolic_Statistics_with_SymPy/}{\textbf{Symbolic
  statistical modeling}}: Adding statistical operations to complex
  physical models.
\end{itemize}
  \end{frame}


\begin{frame}{Features}
  \begin{multicols}{2}
    \tiny
    \begin{itemize}
    \item \textbf{Core Capabilities}
      \begin{itemize}
        \tiny
      \item Basic arithmetic: Support for operators such as +, -, *, /, ** (power)
      \item Simplification
      \item Expansion
      \item Functions: trigonometric, hyperbolic, exponential, roots, logarithms,
        absolute value, spherical harmonics, factorials and gamma functions, zeta
        functions, polynomials, special functions, \ldots
      \item Substitution
      \item Numbers: arbitrary precision integers, rationals, and floats
      \item Noncommutative symbols
      \item Pattern matching
      \end{itemize}
    \item \textbf{Polynomials}
      \begin{itemize}
        \tiny
      \item Basic arithmetic: division, gcd, \ldots
      \item Factorization
      \item Square-free decomposition
      \item Gröbner bases
      \item Partial fraction decomposition
      \item Resultants
      \end{itemize}
    \item \textbf{Calculus}
      \begin{itemize}
        \tiny
      \item Limits: $\lim_{x\to 0}{x\log(x)} = 0$
      \item Differentiation
      \item Integration: It uses extended Risch-Norman heuristic
      \item Taylor (Laurent) series
      \end{itemize}
    \item \textbf{Solving equations}
      \begin{itemize}
        \tiny
      \item Polynomial equations
      \item Algebraic equations
      \item Differential equations
      \item Difference equations
      \item Systems of equations
      \end{itemize}
    \item \textbf{Combinatorics}
      \begin{itemize}
        \tiny
      \item Permutations
      \item Combinations
      \item Partitions
      \item Subsets
      \item Permutation Groups: Polyhedral, Rubik, Symmetric, \ldots
      \item Prufer and Gray Codes
      \end{itemize}

    \end{itemize}
  \end{multicols}
\end{frame}

\begin{frame}{Features}
  \begin{multicols}{2}
    \begin{itemize}
      \tiny
    \item \textbf{Discrete math}
      \begin{itemize}
        \tiny
      \item Binomial coefficients
      \item Summations
      \item Products
      \item Number theory: generating prime numbers, primality testing, integer
        factorization, \ldots
      \item Logic expressions
      \end{itemize}

    \item \textbf{Matrices}
      \begin{itemize}
        \tiny
      \item Basic arithmetic
      \item Eigenvalues/eigenvectors
      \item Determinants
      \item Inversion
      \item Solving
      \item Abstract expressions
      \end{itemize}


    \item \textbf{Geometric Algebra}


    \item \textbf{Geometry}
      \begin{itemize}
        \tiny
      \item points, lines, rays, segments, ellipses, circles, polygons, \ldots
      \item Intersection
      \item Tangency
      \item Similarity
      \end{itemize}

    \item \textbf{Plotting}
      \begin{itemize}
        \tiny
      \item Coordinate modes
      \item Plotting Geometric Entities
      \item 2D and 3D
      \item Interactive interface
      \item Colors
      \end{itemize}

    \item \textbf{Physics}
      \begin{itemize}
        \tiny
      \item Units
      \item Mechanics
      \item Quantum
      \item Gaussian Optics
      \item Pauli Algebra
      \end{itemize}

    \item \textbf{Statistics}
      \begin{itemize}
        \tiny
      \item Normal distributions
      \item Uniform distributions
      \item Probability
      \end{itemize}

    \item \textbf{Printing}
      \begin{itemize}
        \tiny
      \item Pretty printing: ASCII/Unicode pretty printing, LaTeX
      \item Code generation: C, Fortran, Python
      \end{itemize}
    \end{itemize}
  \end{multicols}
\end{frame}


\begin{frame}
\Large Let's try it
\end{frame}

\begin{frame}{Conclusion}
\begin{block}{Recommendations}
\begin{itemize}
\item Try SymPy yourself
\item If you have a problem, consult the documentation \url{http://docs.sympy.org} and contact the community via mailing list or Stack Overflow 
\item If you find an issue, report it
\item SymPy developers will help you to fix it
\item Get involved in SymPy development. It is easy also thanks to
    \url{https://github.com/sympy/sympy/wiki/Development-workflow}
\end{itemize}
\end{block}
\end{frame}




\input{authors.tex}

\end{document}
