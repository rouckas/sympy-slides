\documentclass[aspectratio=169,xcolor=svgnames]{beamer}
\usetheme{Torino}

\usepackage{epsfig} %for figures
\usepackage{xcolor} %for color
\usepackage[utf8]{inputenc}
\usepackage{multicol}
\usepackage{hyperref}

% latex definitions:
\def\d{{\rm d}}
\def\half{{\textstyle{1\over2}}}



\title[SymPy\hspace{4em}\insertframenumber/
\inserttotalframenumber]{~\\ Symbolic computing with SymPy \\~}


\author[Š. Roučka]
{Štěpán Roučka}

\pgfdeclareimage[height=1.5cm]{mylogo}{sympy-250px}
\institute{\pgfuseimage{mylogo}}

\date{October 18, 2018}

\begin{document}

\begin{frame}
  \maketitle
\begin{center}
\normalsize Materials will be available at \url{http://github.com/rouckas/sympy-slides/}
\end{center}
\end{frame}


\begin{frame}{What is SymPy}
  \begin{block}{Goal}
    Provide a symbolic manipulation library in Python.
  \end{block}
  \begin{block}

    ``SymPy is an open source Python library for symbolic mathematics. It aims to
    become a full-featured computer algebra system (CAS) while keeping the code as
    simple as possible in order to be comprehensible and easily extensible. SymPy
      is written entirely in Python and has only one dependence (mpmath).''

  \end{block}
\end{frame}

\begin{frame}{Past and present}
  \begin{block}{History}
    \begin{itemize}
    \item Ondřej Čertík started the project in 2005.
    \item Development took off in 2007 when SymPy first participated in Google
      Summer of Code. We have participated in every Google Summer of Code since.
    \item In 2011, Aaron Meurer (who also joined from Google Summer of Code) took
      over as lead developer.
    \end{itemize}
  \end{block}
  \begin{block}{Current Status}
    \begin{itemize}
    \item Over 750 contributors.
    \item Current code base has over 500,000 lines of code and documentation.
    \end{itemize}
  \end{block}
\end{frame}

\begin{frame}{Why SymPy?}
    \begin{columns}
        \column{0.5\textwidth}
  \begin{block}{}
    \begin{itemize}
      \item Standalone
      \item Full featured
      \item BSD licensed
      \item Embraces Python
      \item Usable as a library
    \end{itemize}
  \end{block}
        \column{0.5\textwidth}
  \begin{block}{Why SymPy for data scientists?}
      It is up to you
  \end{block}
  \begin{block}{Research Articles citing SymPy}
    \begin{itemize}
        \item \href{https://scholar.google.cz/scholar?cites=2023909674275426142&as_sdt=2005&sciodt=0,5&hl=en}{Google Scholar list}
        \item \href{https://www.zotero.org/groups/525293/sympy/items/}{Zotero group}
        \item \href{https://peerj.com/articles/cs-103/}{Sympy paper}
    \end{itemize}
  \end{block}
    \end{columns}
\end{frame}
\begin{frame}{Projects Using SymPy}
    \begin{itemize}
\item
  \href{http://www.sagemath.org/}{\textbf{Sage}}: A CAS, which includes many open source mathematical libraries.
\item
  \href{https://cloud.sagemath.com}{\textbf{SageMathCloud}}:
  A web-based cloud computing and course management
  platform for computational mathematics.
\item
  \href{http://mathpix.com/}{\textbf{Mathpix}}: An iOS App, that detects
  handwritten math as input, and uses SymPy to evaluate it
  and generate the relevant steps to solve the problem.
\item
  \href{http://www.pydy.org/}{\textbf{PyDy}}: Multibody Dynamics with
  Python.
\item
  \href{http://openrave.org/docs/0.8.2/openravepy/ikfast/}{\textbf{IKFast}}:
  IKFast is a robot kinematics compiler provided by
  \href{http://openrave.org/}{OpenRAVE}. It analytically solves robot inverse
  kinematics equations and generates optimized C++. 
\item
  \href{http://octave.sourceforge.net/symbolic/}{\textbf{Octave Symbolic}}:
  Symbolic package adding symbolic calculation features
  to GNU Octave.
\item
  \href{https://github.com/brombo/galgebra}{\textbf{galgebra}}:
  Geometric algebra (previously \texttt{sympy.galgebra}).
    \end{itemize}
\end{frame}
\begin{frame}{Projects Using SymPy}
\begin{itemize}
\item
  \href{https://github.com/jverzani/SymPy.jl}{\textbf{SymPy.jl}}:
  Provides a Julia interface to SymPy using PyCall.
\item
  \href{https://mathics.github.io/}{\textbf{Mathics}}: A
  free, general-purpose online CAS featuring Mathematica compatible
  syntax and functions.
\item
  \href{http://sfepy.org/}{\textbf{SfePy}}: Simple finite elements in
  Python.
\item
  \href{http://quameon.sourceforge.net/}{\textbf{Quameon}}: Quantum
  Monte Carlo in Python.
\item
  \href{http://lcapy.elec.canterbury.ac.nz/}{\textbf{Lcapy}}:
  Experimental Python package for teaching linear circuit analysis.
\item
  \href{http://digitalcommons.calpoly.edu/cgi/viewcontent.cgi?article=1072\&context=physsp/}{\textbf{Quantum
  Programming in Python}}: Quantum 1D Simple Harmonic Oscillator and
  Quantum Mapping Gate.
\item
  \href{http://mech.fsv.cvut.cz/~stransky/software/latexexpr/doc/}{\textbf{LaTeX
  Expression project}}: Easy \LaTeX{} typesetting of algebraic expressions
  in symbolic form with automatic substitution and result computation.
\item
  \href{https://www.researchgate.net/publication/260585491_Symbolic_Statistics_with_SymPy/}{\textbf{Symbolic
  statistical modeling}}: Adding statistical operations to complex
  physical models.
\end{itemize}
  \end{frame}


\begin{frame}{Features}
  \begin{multicols}{2}
    \tiny
    \begin{itemize}
    \item \textbf{Core Capabilities}
      \begin{itemize}
        \tiny
      \item Basic arithmetic: Support for operators such as +, -, *, /, ** (power)
      \item Simplification
      \item Expansion
      \item Functions: trigonometric, hyperbolic, exponential, roots, logarithms,
        absolute value, spherical harmonics, factorials and gamma functions, zeta
        functions, polynomials, special functions, \ldots
      \item Substitution
      \item Numbers: arbitrary precision integers, rationals, and floats
      \item Noncommutative symbols
      \item Pattern matching
      \end{itemize}
    \item \textbf{Polynomials}
      \begin{itemize}
        \tiny
      \item Basic arithmetic: division, gcd, \ldots
      \item Factorization
      \item Square-free decomposition
      \item Gröbner bases
      \item Partial fraction decomposition
      \item Resultants
      \end{itemize}
    \item \textbf{Calculus}
      \begin{itemize}
        \tiny
      \item Limits: $\lim_{x\to 0}{x\log(x)} = 0$
      \item Differentiation
      \item Integration: It uses extended Risch-Norman heuristic
      \item Taylor (Laurent) series
      \end{itemize}
    \item \textbf{Solving equations}
      \begin{itemize}
        \tiny
      \item Polynomial equations
      \item Algebraic equations
      \item Differential equations
      \item Difference equations
      \item Systems of equations
      \end{itemize}
    \item \textbf{Combinatorics}
      \begin{itemize}
        \tiny
      \item Permutations
      \item Combinations
      \item Partitions
      \item Subsets
      \item Permutation Groups: Polyhedral, Rubik, Symmetric, \ldots
      \item Prufer and Gray Codes
      \end{itemize}

    \end{itemize}
  \end{multicols}
\end{frame}

\begin{frame}{Features}
  \begin{multicols}{2}
    \begin{itemize}
      \tiny
    \item \textbf{Discrete math}
      \begin{itemize}
        \tiny
      \item Binomial coefficients
      \item Summations
      \item Products
      \item Number theory: generating prime numbers, primality testing, integer
        factorization, \ldots
      \item Logic expressions
      \end{itemize}

    \item \textbf{Matrices}
      \begin{itemize}
        \tiny
      \item Basic arithmetic
      \item Eigenvalues/eigenvectors
      \item Determinants
      \item Inversion
      \item Solving
      \item Abstract expressions
      \end{itemize}


    \item \textbf{Geometric Algebra}


    \item \textbf{Geometry}
      \begin{itemize}
        \tiny
      \item points, lines, rays, segments, ellipses, circles, polygons, \ldots
      \item Intersection
      \item Tangency
      \item Similarity
      \end{itemize}

    \item \textbf{Plotting}
      \begin{itemize}
        \tiny
      \item Coordinate modes
      \item Plotting Geometric Entities
      \item 2D and 3D
      \item Interactive interface
      \item Colors
      \end{itemize}

    \item \textbf{Physics}
      \begin{itemize}
        \tiny
      \item Units
      \item Mechanics
      \item Quantum
      \item Gaussian Optics
      \item Pauli Algebra
      \end{itemize}

    \item \textbf{Statistics}
      \begin{itemize}
        \tiny
      \item Normal distributions
      \item Uniform distributions
      \item Probability
      \end{itemize}

    \item \textbf{Printing}
      \begin{itemize}
        \tiny
      \item Pretty printing: ASCII/Unicode pretty printing, LaTeX
      \item Code generation: C, Fortran, Python
      \end{itemize}
    \end{itemize}
  \end{multicols}
\end{frame}


\begin{frame}
\Large Let's try it
\end{frame}

\begin{frame}{Conclusion}
\begin{block}{Recommendations}
\begin{itemize}
\item Try SymPy yourself
\item If you have a problem, consult the documentation \url{http://docs.sympy.org} and contact the community via mailing list or Stack Overflow 
\item If you find an issue, report it
\item SymPy developers will help you to fix it
\item Get involved in SymPy development. It is easy also thanks to
    \url{https://github.com/sympy/sympy/wiki/Development-workflow}
\end{itemize}
\end{block}
\end{frame}




\begin{frame}{Authors}
\begin{multicols}{5}
\tiny
Ondřej Čertík\\
Fabian Pedregosa\\
Jurjen N.E. Bos\\
Mateusz Paprocki\\
Marc-Etienne M.Leveille\\
Brian Jorgensen\\
Jason Gedge\\
Robert Schwarz\\
Pearu Peterson\\
Fredrik Johansson\\
Chris Wu\\
Ulrich Hecht\\
Goutham Lakshminarayan\\
David Lawrence\\
Jaroslaw Tworek\\
David Marek\\
Bernhard R. Link\\
Andrej Tokarčík\\
Or Dvory\\
Saroj Adhikari\\
Pauli Virtanen\\
Robert Kern\\
James Aspnes\\
Nimish Telang\\
Abderrahim Kitouni\\
Pan Peng\\
Friedrich Hagedorn\\
Elrond der Elbenfuerst\\
Rizgar Mella\\
Felix Kaiser\\
Roberto Nobrega\\
David Roberts\\
Sebastian Krämer\\
Vinzent Steinberg\\
Riccardo Gori\\
Case Van Horsen\\
Stepan Roucka\\
Ali Raza Syed\\
Stefano Maggiolo\\
Robert Cimrman\\
Bastian Weber\\
Sebastian Krause\\
Sebastian Kreft\\
Dan\\
Alan Bromborsky\\
Boris Timokhin\\
Robert\\
Andy R. Terrel\\
Hubert Tsang\\
Konrad Meyer\\
Henrik Johansson\\
Priit Laes\\
Freddie Witherden\\
Brian E. Granger\\
Andrew Straw\\
Kaifeng Zhu\\
Ted Horst\\
Andrew Docherty\\
Akshay Srinivasan\\
Aaron Meurer\\
Barry Wardell\\
Tomasz Buchert\\
Vinay Kumar\\
Johann Cohen-Tanugi\\
Jochen Voss\\
Luke Peterson\\
Chris Smith\\
Thomas Sidoti\\
Florian Mickler\\
Nicolas Pourcelot\\
Ben Goodrich\\
Toon Verstraelen\\
Ronan Lamy\\
James Abbatiello\\
Ryan Krauss\\
Bill Flynn\\
Kevin Goodsell\\
Jorn Baayen\\
Eh Tan\\
Renato Coutinho\\
Oscar Benjamin\\
Øyvind Jensen\\
Julio Idichekop Filho\\
Łukasz Pankowski\\
Chu-Ching Huang\\
Fernando Perez\\
Raffaele De Feo\\
Christian Muise\\
Matt Curry\\
Kazuo Thow\\
Christian Schubert\\
Jezreel Ng\\
James Pearson\\
Matthew Brett\\
Addison Cugini\\
Nicholas J.S. Kinar\\
Harold Erbin\\
Thomas Dixon\\
Cristóvão Sousa\\
Andre de Fortier Smit\\
Mark Dewing\\
Alexey U. Gudchenko\\
Gary Kerr\\
Sherjil Ozair\\
Oleksandr Gituliar\\
Sean Vig\\
Prafullkumar P. Tale\\
Vladimir Perić\\
Tom Bachmann\\
Yuri Karadzhov\\
\end{multicols}
\end{frame}
\begin{frame}{Authors (continued)}
\begin{multicols}{5}
\tiny
Vladimir Lagunov\\
Matthew Rocklin\\
Saptarshi Mandal\\
Gilbert Gede\\
Anatolii Koval\\
Tomo Lazovich\\
Pavel Fedotov\\
Jack McCaffery\\
Jeremias Yehdegho\\
Kibeom Kim\\
Gregory Ksionda\\
Tomáš Bambas\\
Raymond Wong\\
Luca Weihs\\
Shai 'Deshe' Wyborski\\
Thomas Wiecki\\
Óscar Nájera\\
Mario Pernici\\
Benjamin McDonald\\
Sam Magura\\
Stefan Krastanov\\
Bradley Froehle\\
Min Ragan-Kelley\\
Emma Hogan\\
Nikhil Sarda\\
Julien Rioux\\
Roberto Colistete, Jr.\\
Raoul Bourquin\\
Gert-Ludwig Ingold\\
Srinivas Vasudevan\\
Jason Moore\\
Miha Marolt\\
Tim Lahey\\
Luis Garcia\\
Matt Rajca\\
David Li\\
Alexandr Gudulin\\
Bilal Akhtar\\
Grzegorz Świrski\\
Matt Habel\\
David Ju\\
Nichita Utiu\\
Nikolay Lazarov\\
Steve Anton\\
Imran Ahmed Manzoor\\
Ljubiša Moćić\\
Piotr Korgul\\
Jim Zhang\\
Sam Sleight\\
tborisova\\
Chancellor Arkantos\\
Stepan Simsa\\
Tobias Lenz\\
Siddhanathan Shanmugam\\
Tiffany Zhu\\
Tristan Hume\\
Alexey Subach\\
Joan Creus\\
Geoffry Song\\
Puneeth Chaganti\\
Marcin Kostrzewa\\
Natalia Nawara\\
vishal\\
Shruti Mangipudi\\
Davy Mao\\
Swapnil Agarwal\\
Dhia Kennouche\\
jerryma1121\\
Joachim Durchholz\\
Martin Povišer\\
Siddhant Jain\\
Kevin Hunter\\
Michael Mayorov\\
Nathan Alison\\
Christian Bühler\\
Carsten Knoll\\
Bharath M R\\
Matthias Toews\\
Sergiu Ivanov\\
Jorge E. Cardona\\
Sanket Agarwal\\
Manoj Babu K.\\
Sai Nikhil\\
Aleksandar Makelov\\
Sachin Irukula\\
Raphael Michel\\
Ashwini Oruganti\\
Andreas Kloeckner\\
Prateek Papriwal\\
Arpit Goyal\\
Angadh Nanjangud\\
Comer Duncan\\
Jens H. Nielsen\\
Joseph Dougherty\\
Elliot Marshall\\
Guru Devanla\\
George Waksman\\
Alexandr Popov\\
Tarun Gaba\\
Takafumi Arakaki\\
Saurabh Jha\\
Rom le Clair\\
Angus Griffith\\
Timothy Reluga\\
Brian Stephanik\\
Alexander Eberspächer\\
Sachin Joglekar\\
Tyler Pirtle\\
Vasily Povalyaev\\
Colleen Lee\\
\end{multicols}
\end{frame}
\begin{frame}{Authors (continued)}
\begin{multicols}{5}
\tiny
Matthew Hoff\\
Niklas Thörne\\
Huijun Mai\\
Marek Šuppa\\
Ramana Venkata\\
Prasoon Shukla\\
Stefen Yin\\
Thomas Hisch\\
Madeleine Ball\\
Mary Clark\\
Rishabh Dixit\\
Manoj Kumar\\
Akshit Agarwal\\
CJ Carey\\
Patrick Lacasse\\
Ananya H\\
Tarang Patel\\
Christopher Dembia\\
Benjamin Fishbein\\
Sean Ge\\
Amit Jamadagni\\
Ankit Agrawal\\
Björn Dahlgren\\
Christophe Saint-Jean\\
Demian Wassermann\\
Khagesh Patel\\
Stephen Loo\\
hm\\
Patrick Poitras\\
Katja Sophie Hotz\\
Varun Joshi\\
Chetna Gupta\\
Thilina Rathnayake\\
Max Hutchinson\\
Shravas K Rao\\
Matthew Tadd\\
Alexander Hirzel\\
Randy Heydon\\
Oliver Lee\\
Seshagiri Prabhu\\
Pradyumna\\
Erik Welch\\
Eric Nelson\\
Roland Puntaier\\
Chris Conley\\
Tim Swast\\
Dmitry Batkovich\\
Francesco Bonazzi\\
Yuriy Demidov\\
Rick Muller\\
Manish Gill\\
Markus Müller\\
Amit Saha\\
Jeremy\\
QuaBoo\\
Stefan van der Walt\\
David Joyner\\
Lars Buitinck\\
Alkiviadis G. Akritas\\
Vinit Ravishankar\\
Mike Boyle\\
Heiner Kirchhoffer\\
Pablo Puente\\
James Fiedler\\
Harsh Gupta\\
Tuomas Airaksinen\\
Paul Strickland\\
James Goppert\\
rathmann\\
Avichal Dayal\\
Paul Scott\\
Shipra Banga\\
Pramod Ch\\
Akshay\\
Buck Shlegeris\\
Jonathan Miller\\
Edward Schembor\\
Rajath Shashidhara\\
Zamrath Nizam\\
Aditya Shah\\
Rajat Aggarwal\\
Sambuddha Basu\\
Zeel Shah\\
Abhinav Chanda\\
Jim Crist\\
Sudhanshu Mishra\\
Anurag Sharma\\
Soumya Dipta Biswas\\
Sushant Hiray\\
Ben Lucato\\
Kunal Arora\\
Henry Gebhardt\\
Dammina Sahabandu\\
Manish Shukla\\
Ralph Bean\\
richierichrawr\\
John Connor\\
Juan Luis Cano Rodríguez\\
Sahil Shekhawat\\
Kundan Kumar\\
Stas Kelvich\\
sevaader\\
Dhruvesh Vijay Parikh\\
Venkatesh Halli\\
Lennart Fricke\\
Vlad Seghete\\
Shashank Agarwal\\
carstimon\\
Pierre Haessig\\
Maciej Baranski\\
\end{multicols}
\end{frame}
\begin{frame}{Authors (continued)}
\begin{multicols}{5}
\tiny
Benjamin Gudehus\\
Faisal Anees\\
Mark Shoulson\\
Robert Johansson\\
Kalevi Suominen\\
Kaushik Varanasi\\
Fawaz Alazemi\\
Ambar Mehrotra\\
David P. Sanders\\
Peter Brady\\
John V. Siratt\\
Sarwar Chahal\\
Nathan Woods\\
Colin B. Macdonald\\
Marcus Näslund\\
Clemens Novak\\
Mridul Seth\\
Craig A. Stoudt\\
Raj\\
Mihai A. Ionescu\\
immerrr\\
Chai Wah Wu\\
Leonid Blouvshtein\\
Peleg Michaeli\\
ck Lux\\
zsc347\\
Hamish Dickson\\
Michael Gallaspy\\
Roman Inflianskas\\
Duane Nykamp\\
Ted Dokos\\
Sunny Aggarwal\\
Victor Brebenar\\
Akshat Jain\\
Shivam Vats\\
Longqi Wang\\
Juan Felipe Osorio\\
Ray Cathcart\\
Lukas Zorich\\
Eric Miller\\
Cody Herbst\\
Nishith Shah\\
Amit Kumar\\
Yury G. Kudryashov\\
Guillaume Gay\\
Mihir Wadwekar\\
Tuan Manh Lai\\
Asish Panda\\
Darshan Chaudhary\\
Alec Kalinin\\
Ralf Stephan\\
Aaditya Nair\\
Jayesh Lahori\\
Harshil Goel\\
Luv Agarwal\\
Jason Ly\\
Lokesh Sharma\\
Sartaj Singh\\
Chris Swierczewski\\
Konstantin Togoi\\
Param Singh\\
Sumith Kulal\\
Juha Remes\\
Philippe Bouafia\\
Peter Schmidt\\
Jiaxing Liang\\
Lucas Jones\\
Gregory Ashton\\
Jennifer White\\
Renato Orsino\\
Michael Boyle\\
Alistair Lynn\\
Govind Sahai\\
Adam Bloomston\\
Kyle McDaniel\\
Nguyen Truong Duy\\
Alex Lindsay\\
Mathew Chong\\
Jason Siefken\\
Gaurav Dhingra\\
Gao, Xiang\\
Kevin Ventullo\\
mao8\\
Isuru Fernando\\
Shivam Tyagi\\
Richard Otis\\
Rich LaSota\\
dustyrockpyle\\
Anton Akhmerov\\
Michael Zingale\\
Chak-Pong Chung\\
David T\\
Phil Ruffwind\\
Sebastian Koslowski\\
Kumar Krishna Agrawal\\
Dustin Gadal\\
João Moura\\
Yu Kobayashi\\
Shashank Kumar\\
Timothy Cyrus\\
Devyani Kota\\
Keval Shah\\
Dzhelil Rufat\\
Pastafarianist\\
Sourav Singh\\
Jacob Garber\\
Vinay Singh\\
GolimarOurHero\\
Prashant Tyagi\\
Matthew Davis\\
\end{multicols}
\end{frame}
\begin{frame}{Authors (continued)}
\begin{multicols}{5}
\tiny
Tschijnmo TSCHAU\\
Alexander Bentkamp\\
Jack Kemp\\
Kshitij Saraogi\\
Thomas Baruchel\\
Nicolás Guarín-Zapata\\
Jens Jørgen Mortensen\\
Sampad Kumar Saha\\
Eva Charlotte Mayer\\
Laura Domine\\
Justin Blythe\\
Meghana Madhyastha\\
Tanu Hari Dixit\\
Shekhar Prasad Rajak\\
Aqnouch Mohammed\\
Arafat Dad Khan\\
Boris Atamanovskiy\\
Sam Tygier\\
Jai Luthra\\
Guo Xingjian\\
Sandeep Veethu\\
Archit Verma\\
Shubham Tibra\\
Ashutosh Saboo\\
Michael S. Hansen\\
Anish Shah\\
Guillaume Jacquenot\\
Bhautik Mavani\\
Michał Radwański\\
Jerry Li\\
Pablo Zubieta\\
Shivam Agarwal\\
Chaitanya Sai Alaparthi\\
Arihant Parsoya\\
Ruslan Pisarev\\
Akash Trehan\\
Nishant Nikhil\\
Vladimir Poluhsin\\
Akshay Nagar\\
James Brandon Milam\\
Abhinav Agarwal\\
Rishabh Daal\\
Sanya Khurana\\
Aman Deep\\
Aravind Reddy\\
Abhishek Verma\\
Matthew Parnell\\
Thomas Hickman\\
Akshay Siramdas\\
YiDing Jiang\\
Jatin Yadav\\
Matthew Thomas\\
Rehas Sachdeva\\
Michael Mueller\\
Srajan Garg\\
Prabhjot Singh\\
Haruki Moriguchi\\
Tom Gijselinck\\
Nitin Chaudhary\\
Alex Argunov\\
Nathan Musoke\\
Abhishek Garg\\
Dana Jacobsen\\
Vasiliy Dommes\\
Phillip Berndt\\
Haimo Zhang\\
Anthony Scopatz\\
bluebrook\\
Leonid Kovalev\\
Josh Burkart\\
Dimitra Konomi\\
Christina Zografou\\
Fiach Antaw\\
Langston Barrett\\
Krit Karan\\
G. D. McBain\\
Prempal Singh\\
Gabriel Orisaka\\
Matthias Bussonnier\\
rahuldan\\
Colin Marquardt\\
Andrew Taber\\
Yash Reddy\\
Peter Stangl\\
elvis-sik\\
Nikos Karagiannakis\\
Jainul Vaghasia\\
Dennis Meckel\\
Harshil Meena\\
Micky\\
Nick Curtis\\
Michele Zaffalon\\
Martha Giannoudovardi\\
Devang Kulshreshtha\\
Steph Papanik\\
Mohammad Sadeq Dousti\\
Arif Ahmed\\
Abdullah Javed Nesar\\
Lakshya Agrawal\\
shruti\\
Rohit Rango\\
Hong Xu\\
Ivan Petuhov\\
Alsheh\\
Marcel Stimberg\\
Alexey Pakhocmhik\\
Tommy Olofsson\\
Zulfikar\\
Blair Azzopardi\\
Danny Hermes\\
\end{multicols}
\end{frame}
\begin{frame}{Authors (continued)}
\begin{multicols}{5}
\tiny
Sergey Pestov\\
Mohit Chandra\\
Karthik Chintapalli\\
Marcin Briański\\
andreo\\
Flamy Owl\\
Yicong Guo\\
Varun Garg\\
Rishabh Madan\\
Aditya Kapoor\\
Karan Sharma\\
Vedant Rathore\\
Johan Blåbäck\\
Pranjal Tale\\
Jason Tokayer\\
Raghav Jajodia\\
Rajat Thakur\\
Dhruv Bhanushali\\
Anjul Kumar Tyagi\\
Barun Parruck\\
Bao Chau\\
Tanay Agrawal\\
Ranjith Kumar\\
Shikhar Makhija\\
Yathartha Joshi\\
Valeriia Gladkova\\
Sagar Bharadwaj\\
Daniel Mahler\\
Ka Yi\\
Rishat Iskhakov\\
Szymon Mieszczak\\
Sachin Agarwal\\
Priyank Patel\\
Satya Prakash Dwibedi\\
tools4origins\\
Nico Schlömer\\
Fermi Paradox\\
Ekansh Purohit\\
Vedarth Sharma\\
Peeyush Kushwaha\\
Jayjayyy\\
Christopher J. Wright\\
Jakub Wilk\\
Mauro Garavello\\
Chris Tefer\\
Shikhar Jaiswal\\
Chiu-Hsiang Hsu\\
Carlos Cordoba\\
Fabian Ball\\
Yerniyaz\\
Christiano Anderson\\
Robin Neatherway\\
Thomas Hunt\\
Theodore Han\\
Duc-Minh Phan\\
Lejla Metohajrova\\
Samyak Jain\\
Aditya Rohan\\
Vincent Delecroix\\
Michael Sparapany\\
Harsh Jain\\
Nathan Goldbaum\\
latot\\
Kenneth Lyons\\
Stan Schymanski\\
David Daly\\
Ayush Shridhar\\
Javed Nissar\\
Jiri Kuncar\\
vedantc98\\
Rupesh Harode\\
Rob Zinkov\\
James Harrop\\
James Taylor\\
Ishan Joshi\\
Marco Mancini\\
Boris Ettinger\\
Micah Fitch\\
Daniel Wennberg\\
ylemkimon\\
Akash Vaish\\
Peter Enenkel\\
Waldir Pimenta\\
Jithin D. George\\
Lev Chelyadinov\\
Lucas Wiman\\
Rhea Parekh\\
James Cotton\\
Robert Pollak\\
anca-mc\\
Sourav Ghosh\\
Jonathan Allan\\
Nikhil Pappu\\
Ethan Ward\\
Cezary Marczak\\
dps7ud\\
Nilabja Bhattacharya\\
Itay4\\
Poom Chiarawongse\\
Yang Yang\\
Cavendish McKay\\
Bradley Gannon\\
B McG\\
Rob Drynkin\\
Seth Ebner\\
Akash Kundu\\
Mark Jeromin\\
Roberto Díaz Pérez\\
Gleb Siroki\\
Segev Finer\\
\end{multicols}
\end{frame}
\begin{frame}{Authors (continued)}
\begin{multicols}{5}
\tiny
Alex Lubbock\\
Ayodeji Ige\\
Matthew Wardrop\\
Hugo\\
Austin Palmer\\
der-blaue-elefant\\
Filip Gokstorp\\
Yuki Matsuda\\
Aaron Miller\\
Salil Vishnu Kapur\\
Atharva Khare\\
Shubham Maheshwari\\
Pavel Tkachenko\\
Ashish Kumar Gaurav\\
Rajeev Singh\\
Keno Goertz\\
Lucas Gallindo\\
Himanshu\\
David Menéndez Hurtado\\
Amit Manchanda\\
Rohit Jain\\
Jonathan A. Gross\\
Unknown\\
Sayan Goswami\\
Subhash Saurabh\\
Rastislav Rabatin\\
Vishal\\
Jeremey Gluck\\
Akshat Maheshwari\\
symbolique\\
Saloni Jain\\
Arighna Chakrabarty\\
Abhigyan Khaund\\
Jashanpreet Singh\\
Saurabh Agarwal\\
luz.paz\\
P. Sai Prasanth\\
Nirmal Sarswat\\
Cristian Di Pietrantonio\\
Ravi charan\\
Nityananda Gohain\\
Cédric Travelletti\\
Nicholas Bollweg\\
Himanshu Ladia\\
eward\\
Adwait Baokar\\
Mihail Tarigradschi\\
Saketh\\
rushyam\\
sfoo\\
Rahil Hastu\\
Zach Raines\\
Sidhant Nagpal\\
Gagandeep Singh\\
Rishav Chakraborty\\
Malkhan Singh\\
Joaquim Monserrat\\
Mayank Singh\\
Rémy Léone\\
Maxence Mayrand\\
Nikoleta Glynatsi\\
helo9\\
Ken Wakita\\
Carl Sandrock\\
Fredrik Eriksson\\
Ian Swire\\
Bulat\\
Ehren Metcalfe\\
Dmitry Savransky\\
Kiyohito Yamazaki\\
Caley Finn\\
zhouzq-thu\\
Alexander Pozdneev\\
Wes Turner\\
JMSS-Unknown\\
Arshdeep Singh\\
cym1\\
Stewart Wadsworth\\
Jared Lumpe\\
Avi Shrivastava\\
ramvenkat98\\
Bilal Ahmed\\
Dimas Abreu Archanjo Dutra\\
Yatna Verma\\
S.Y. Lee\\
Miro Hrončok\\
Sudarshan Kamath\\
Ayushman Koul\\
Robert Dougherty-Bliss\\
Andrey Grozin\\
Bavish Kulur\\
\end{multicols}
\end{frame}


\end{document}
